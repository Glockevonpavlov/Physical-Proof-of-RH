% ============================================================

% Unified Paper: The Riemann--Pavlov Operator

% One-file Overleaf-ready LaTeX source

% ============================================================
\documentclass[11pt]{article}

% ---------- Packages ----------
\usepackage[a4paper,margin=1in]{geometry}
\usepackage{amsmath,amssymb,amsthm,mathtools}
\usepackage{bm}
\usepackage{physics}
\usepackage{hyperref}
\usepackage{microtype}
\usepackage{graphicx}
\usepackage{caption}
\usepackage{subcaption}

% Hyperref anchor stability
\makeatletter
\renewcommand*{\theHequation}{\thesection.\arabic{equation}}
\makeatother



% ---------- Theorem environments ----------

\newtheorem{theorem}{Theorem}[section]
\newtheorem{lemma}[theorem]{Lemma}
\newtheorem{proposition}[theorem]{Proposition}
\newtheorem{corollary}[theorem]{Corollary}
\theoremstyle{definition}
\newtheorem{definition}[theorem]{Definition}
\newtheorem{axiom}[theorem]{Axiom}
\newtheorem{remark}[theorem]{Remark}
\newtheorem{conjecture}[theorem]{Conjecture}

% ---------- Notation ----------

\newcommand{\R}{\mathbb R}
\newcommand{\N}{\mathbb N}
\newcommand{\xhat}{\hat x}
\newcommand{\phat}{\hat p}
\newcommand{\Hzero}{\hat H_0}
\newcommand{\Hrp}{\hat H_{\mathrm{RP}}}
\newcommand{\Htil}{\widetilde H}
\newcommand{\etaPT}{\eta_{\mathrm{PT}}}
\newcommand{\rhoPT}{\rho}
\newcommand{\wconf}{w_{\mathrm{conf}}}

% \erf is defined by some packages; use \operatorname{erf} directly

\newcommand{\SiF}{\operatorname{Si}}
\newcommand{\XiFun}{\Xi}

% ============================================================

\begin{document}

\title{The Riemann--Pavlov Operator:\\
PT Metric, Canonical Trace, and Determinant-Level Realization of the Completed Zeta Function}
\author{Donghwi Seo \& CosmosT}
\date{\today}
\maketitle

\begin{abstract}
We present a non-Hermitian dilation-based operator model whose kinematics is designed to reproduce the arithmetic
time-support $t=\log n$ and the von Mangoldt weights $\Lambda(n)$ at the level of an exact distributional trace.
The model combines (i) a PT-metric $\eta_{\mathrm{PT}}$ derived from a first-order similarity equation,
(ii) a confining \emph{state-space} metric (\emph{CP-stiffness}) implemented as a weighted physical domain,
and (iii) an infrared (IR) topological regulator in log-coordinates to define zeta-regularized determinants.
Under canonicality requirements (symmetry, archimedean matching, locality of counterterms, normalization),
the normalized determinant associated with the Hermitian representative $\widetilde H$ is forced to coincide with
$\Xi(E)/\Xi(0)$, where $\Xi$ is the completed Riemann zeta function on the critical line.
We also include numerical signatures (GUE spacing, spectral form factor, Berry-phase locking, and Stokes turning points)
as consistency evidence near the critical coupling $\epsilon=2.5$.
\end{abstract}

\tableofcontents

% ============================================================

\section{Introduction}
The Hilbert--Pólya philosophy suggests that the nontrivial zeros of the Riemann zeta function may arise as a spectrum of
a suitable (self-adjoint) operator. A persistent obstacle in dilation-based models (e.g.\ Berry--Keating type)
is the \emph{continuum leakage}: on $L^2(\R)$ the natural dilation generator has continuous spectrum, and naive
potential deformations do not automatically discretize it without breaking the arithmetic structure.

This paper adopts a different division of labor:

\begin{itemize}
\item \textbf{Dynamics} is kept \emph{exactly} dilation-linear in log-coordinates so that arithmetic periods
$T_n=\log n$ appear as atomic supports in time.
\item \textbf{Observability} (real spectrum / unitary evolution) is enforced by a derived PT-metric $\eta_{\mathrm{PT}}$.
\item \textbf{Discreteness and determinant regularity} are implemented by an \emph{IR topological regulator} together with a
\emph{state-space} confining metric (\emph{CP-stiffness}) which selects physical states and renders trace functionals finite.
\end{itemize}

The outcome is a determinant-level identification with the completed zeta function $\Xi(E)$, formulated as a
canonical reduction: once the kinematics and canonicality conditions are fixed, the resulting normalized determinant is forced to match
$\Xi(E)/\Xi(0)$.

% ============================================================

\section{Model, Physical Domain, and IR Regulator}
\subsection{Gamma representation (archimedean motif)}
We recall the standard Gaussian form of the gamma factor:
\begin{equation}
\Gamma\!\Big(\frac{s}{2}\Big)=2\int_{0}^{\infty} x^{s-1}e^{-x^2}\,dx .
\label{eq:gamma_gauss}
\end{equation}

This motivates a Gaussian-type confinement at the level of the \emph{physical domain} rather than by modifying the kinematic generator.
\subsection{Riemann--Pavlov operator}

Let $p:=-i\frac{d}{dx}$ and let $x$ denote multiplication on $L^2(\R,dx)$.

Define the symmetric dilation generator

\begin{equation}
\Hzero:=\frac12(\xhat\phat+\phat\xhat)=-i\Big(x\frac{d}{dx}+\frac12\Big).
\label{eq:H0_def}
\end{equation}

Fix real parameters $\lambda\in\R$ and $\epsilon\in\R$ and define

\begin{equation}
\Hrp:=\Hzero+i\lambda f(x),
\qquad
f(x):=x e^{-x^2}+\epsilon \sin x .
\label{eq:HRP_def}
\end{equation}

The operator $\Hrp$ is PT-symmetric in the usual sense (multiplication by an odd real function in the imaginary part).



\subsection{CP-stiffness as a state-space metric (not a potential)}
The confining mechanism is introduced as a \emph{metric on the state space}:
\begin{equation}
\wconf(x):=e^{\mu x^2},
\qquad
\mu>0 .
\label{eq:wconf_def}
\end{equation}

We define the physical domain

\begin{equation}
\mathcal D_{\mathrm{phys}}
:=
\Big\{
\psi\in L^2(\R)\,:\,
\psi(-x)=-\psi(x),
\ \int_{\R}|\psi(x)|^2\,\wconf(x)\,dx<\infty
\Big\}.
\label{eq:Dphys_def}
\end{equation}


% [Section 2.3 Remark Modification]

\begin{remark}[Stiffness Hypothesis]
We \textit{hypothesize} that requiring the physical ground state to match the self-dual Riemann kernel fixes the stiffness to $\mu = \pi^2/2$. While numerically supported ($>99.9\%$), for the rigorous derivations in this paper, we treat $\mu$ as a free positive parameter ensuring trace convergence.
\end{remark}



\subsection{IR topological regulator in log-coordinates}
To define zeta-regularized determinants we introduce an IR regulator by compactifying log-space to a circle.
After parity projection (Appendix~\ref{app:parity_log}), we work on $x>0$ and use $q=\log x\in\R$.
For $L>0$ and a twist angle $\gamma_B\in\R$ (Berry-phase twist) we impose
\begin{equation}
\phi(q+2L)=e^{i\gamma_B}\phi(q),
\qquad q\in[-L,L] .
\label{eq:twisted_bc}
\end{equation}

This makes the translation generator $-i\partial_q$ have discrete spectrum.
We will define determinants at finite $L$ and extract a canonical $L\to\infty$ limit by a unique subtraction principle (Section~\ref{sec:det_matching}).

% ============================================================

\section{PT Metric and Hermitian Equivalence}
\label{sec:pt_metric}

\subsection{Similarity calculus for multiplication metrics}
Let $\eta=\eta(x)>0$ act by multiplication and be $C^1$.
On a common dense domain stable under multiplication by $\eta^{\pm1}$ one has the standard identities:
\begin{align}
\eta\,p\,\eta^{-1} &= p + i(\ln\eta)' ,
\label{eq:eta_p}\\
\eta\,\Hzero\,\eta^{-1} &= \Hzero + i\,x(\ln\eta)' .
\label{eq:eta_H0}\\
\eta\,\Hrp\,\eta^{-1} &= \Hzero + i\,x(\ln\eta)' + i\lambda f(x).
\label{eq:eta_HRP}
\end{align}

\subsection{ODE for pseudo-Hermiticity and closed form}
Since $\Hzero^\dagger=\Hzero$ on the physical parity sector (Appendix~\ref{app:parity_log}), we have $\Hrp^\dagger=\Hzero-i\lambda f(x)$.
Thus $\Hrp^\dagger=\eta \Hrp\eta^{-1}$ holds iff
\begin{equation}
x(\ln\eta)'=-2\lambda f(x)=-2\lambda\big(xe^{-x^2}+\epsilon\sin x\big)\qquad(x\neq0),
\label{eq:eta_ode}
\end{equation}

with continuous extension at $x=0$.

Integrating~\eqref{eq:eta_ode} yields the PT metric:
\begin{equation}
\etaPT(x)
=
\exp\Big(
-\lambda\sqrt\pi\,\operatorname{erf}(x)-2\lambda\epsilon\,\SiF(x)
\Big).
\label{eq:etaPT_closed}
\end{equation}

\begin{lemma}[Boundedness and bounded invertibility]
\label{lem:eta_bounded}
Using $|\operatorname{erf}(x)|\le 1$ and $|\SiF(x)|\le \pi/2$,
\begin{equation}
e^{-M}\le \etaPT(x)\le e^{M},
\qquad
M:=|\lambda|\sqrt\pi+|\lambda\epsilon|\pi.
\label{eq:eta_bounds}
\end{equation}

Hence $\etaPT$ and $\etaPT^{-1}$ are bounded multiplication operators.
\end{lemma}

\subsection{Hermitian representative}
Define
\begin{equation}
\rhoPT:=\etaPT^{1/2}
=
\exp\Big(
-\tfrac{\lambda}{2}\sqrt\pi\,\operatorname{erf}(x)-\lambda\epsilon\,\SiF(x)
\Big),
\label{eq:rho_def}
\end{equation}
and set $\Htil:=\rhoPT \Hrp \rhoPT^{-1}$.

\begin{theorem}[Hermitian equivalence]
\label{thm:hermitian_equivalence}
On $\mathcal D_{\mathrm{phys}}$ (and its regulated versions on the half-line / circle),
\begin{equation}
\Htil=\Hzero,
\label{eq:Htil_equals_H0}
\end{equation}

and $\Htil$ is self-adjoint on the physical parity sector.
Consequently, $\Hrp$ generates unitary time evolution in the $\etaPT$-inner product.
\end{theorem}

\begin{remark}[Division of roles]
The PT metric $\etaPT$ guarantees \emph{observability} (real spectrum/unitarity). The CP-stiffness weight $\wconf$ is independent of $\etaPT$;
it selects physical states and regularizes traces, without changing the kinematic generator $\Hzero$.
\end{remark}

% ============================================================

\section{Log-Kinematics, Arithmetic Operators, and Atomic Time Support}
\label{sec:arithmetic_trace}

\subsection{Log-coordinate map}
Restrict to $x>0$ and define the unitary map $\mathcal U:L^2(\R_+,dx)\to L^2(\R,dq)$ by
\begin{equation}
(\mathcal U\psi)(q):=e^{q/2}\psi(e^q),
\qquad q:=\log x .
\label{eq:U_log}
\end{equation}

Then
\begin{equation}
\mathcal U\,\Hzero\,\mathcal U^{-1}=-i\partial_q.
\label{eq:H0_to_translation}
\end{equation}

\subsection{Integer dilations}
For $n\in\N^\times$ define the unitary dilation operators on $L^2(\R_+,dx)$ by
\begin{equation}
(V_n\psi)(x):=n^{1/2}\psi(nx).
\label{eq:Vn_def}
\end{equation}

Under $\mathcal U$ these become translations:

\begin{equation}
(\mathcal U V_n\mathcal U^{-1}\phi)(q)=\phi(q+\log n),
\qquad
\langle q|V_n|q'\rangle=\delta(q-q'+\log n).
\label{eq:Vn_kernel}
\end{equation}

Let $U(t):=e^{-it\Htil}=e^{-it\Hzero}$ be the unitary flow. In $q$-space $U(t)$ is translation by $t$.
Hence, for each $n$,
\begin{equation}
\langle q|V_nU(t)|q\rangle=\delta(t-\log n)
\label{eq:atomic_support}
\end{equation}

as an \emph{exact} distributional identity.



% --- [REPLACEMENT for Section 4.3] ---

\subsection{CP-stiffness weighted trace}
To ensure convergence at both asymptotic limits ($q \to \pm\infty$), we adopt a \textbf{symmetric (self-dual) stiffness weight}:
\begin{equation}
\tau_\mu(A):=\int_{-\infty}^{\infty}\langle q|A|q\rangle\, \exp\Big(-\mu(e^{2q} + e^{-2q})\Big)\,dq .
\label{eq:tau_mu_def}
\end{equation}
This weight $w(q) = \exp(-\mu(2\cosh 2q))$ acts as a soft wall at both boundaries, ensuring the trace is finite.

\begin{lemma}[Atomic time support survives CP-stiffness]
\label{lem:atomic_tau}
For each $n\in\N^\times$, the trace localization holds:
\begin{equation}
\tau_\mu(V_nU(t))=C(\mu)\,\delta(t-\log n),
\qquad
C(\mu):=\int_{-\infty}^{\infty} \exp\Big(-\mu(e^{2q} + e^{-2q})\Big)\,dq < \infty.
\label{eq:tau_mu_atomic}
\end{equation}

The normalization constant $C(\mu)$ corresponds to a modified Bessel function of the second kind, $K_0(2\mu)$, which is strictly finite for $\mu > 0$.
\end{lemma}

% -------------------------------------

\subsection{Euler product in an operator algebra}
Let $\mathfrak B$ be the commutative Banach algebra generated (in operator norm) by finite linear combinations of $\{V_n\}$.
For $\Re s>1$ define
\begin{equation}
\mathcal Z(s):=\sum_{n\ge1}n^{-s}V_n\in\mathfrak B.
\label{eq:Zs_def}
\end{equation}

\begin{lemma}[Euler product in $\mathfrak B$]
\label{lem:euler_product}
For $\Re s>1$,
\begin{equation}
\mathcal Z(s)=\prod_{p}\big(1-p^{-s}V_p\big)^{-1}
\label{eq:euler_product}
\end{equation}
as a norm-convergent product.
\end{lemma}

Define the von Mangoldt operator (prime generator) by
\begin{equation}
\mathcal L(s):=-\partial_s\log\mathcal Z(s)
=\sum_{n\ge1}\Lambda(n)n^{-s}V_n
=\sum_p\sum_{m\ge1}(\log p)p^{-ms}V_{p^m}.
\label{eq:L_def}
\end{equation}

\begin{lemma}[Prime-orbit distribution]
\label{lem:prime_distribution}
For $\Re s>1$,
\begin{equation}
\tau_\mu(\mathcal L(s)U(t))
=
C(\mu)\sum_{n\ge1}\Lambda(n)n^{-s}\delta(t-\log n).
\label{eq:prime_dist}
\end{equation}
\end{lemma}

% ============================================================

\section{Canonical Determinant Matching}
\label{sec:det_matching}

\subsection{Target entire function}
Let
\begin{equation}
\xi(s):=\frac12\,s(s-1)\,\pi^{-s/2}\Gamma\!\Big(\frac{s}{2}\Big)\zeta(s),
\qquad
\XiFun(E):=\xi\!\Big(\frac12+iE\Big).
\label{eq:Xi_def}
\end{equation}

Then $\XiFun(E)$ is an even real entire function of order $1$.

\subsection{Regulated determinant}
On the IR-regulated circle $q\in[-L,L]$ with twist~\eqref{eq:twisted_bc}, the operator $\mathcal U\Hzero\mathcal U^{-1}=-i\partial_q$
has discrete spectrum. We define the normalized zeta determinant
\begin{equation}
\Delta_{L}(E):=
\frac{\det_{\zeta}(\Htil_{L}^{\,2}+E^2)}{\det_{\zeta}(\Htil_{L}^{\,2})},
\qquad
\Delta_L(0)=1,
\label{eq:Delta_L}
\end{equation}

where $\Htil_L$ denotes the regulated self-adjoint representative (the restriction of $\Htil$ to the twisted circle).

% --- [REPLACEMENT for Section 5.2 & 5.3] ---

\subsection{Canonicality axioms (determinant normalization)}
\label{sec:canon_axioms}
\begin{axiom}[Canonicality]\label{ax:canonicality}
We require the renormalized determinant $\Delta_{\mathrm{ren}}(s)$ to satisfy:
\begin{enumerate}
  \item[(C1)] \textbf{Reality/Evenness on the critical line:} $\Delta_{\mathrm{ren}}(1/2+iE)\in\mathbb R$ and is even in $E$.
  \item[(C2)] \textbf{Archimedean matching:} the $\Gamma(s/2)$ factor is reproduced in the smooth (Weyl) part.
  \item[(C3)] \textbf{Local counterterms only:} renormalization may multiply the determinant only by $\exp(P(s))$ where $P$ is a polynomial of degree $\le 1$.
  \item[(C4)] \textbf{Normalization:} $\Delta_{\mathrm{ren}}(1/2)=1$ (equivalently $\Delta_{\mathrm{ren}}(0)=1$ in $E$-notation).
\end{enumerate}
\end{axiom}

\subsection{Dynamical determinant from the prime trace}
Let $\bar\tau_\mu := \tau_\mu/C(\mu)$ be the normalized stiffness trace.
For $\Re s>1$ and $\alpha>0$ define
\begin{equation}
\log \Delta_{\mu,\alpha}(s)
:=\int_{0}^{\infty}\frac{dt}{t}\,e^{-\alpha t}\,
\bar\tau_\mu\!\big(\mathcal L(s)U(t)\big).
\label{eq:dyn_det_trace_def}
\end{equation}
Using Lemma~\ref{lem:prime_distribution} and the atomic support $t=\log n$,
\begin{equation}
\log \Delta_{\mu,\alpha}(s)
=\sum_{n\ge 2}\frac{\Lambda(n)}{\log n}\,\frac{1}{n^{s+\alpha}}
=\log \zeta(s+\alpha).
\label{eq:dyn_det_to_logzeta}
\end{equation}
We then define $\Delta_\mu(s)$ by canonical subtraction as $\alpha\downarrow 0$, separating the pole at $s=1$ into the
explicit factor $s(s-1)$ in the completion below.

\subsection{Archimedean completion and canonical closure}
Define the completed (renormalized) determinant by
\begin{equation}
\Delta_{\mathrm{ren}}(s)
:= \frac12\,s(s-1)\,\pi^{-s/2}\Gamma\!\Big(\frac{s}{2}\Big)\,
\Delta_\mu(s)\,\exp(-P(s)),
\label{eq:det_ren_def}
\end{equation}
where $P(s)=a+bs$ is fixed by Axiom~\ref{ax:canonicality}(C1)--(C4).
\begin{theorem}[Canonical Closure]
\label{thm:canonical_closure}
Under Axiom~\ref{ax:canonicality}, we have
\[
\Delta_{\mathrm{ren}}(s)\equiv \frac{\xi(s)}{\xi(1/2)},
\qquad\text{hence}\qquad
\Delta_{\mathrm{ren}}(1/2+iE)=\frac{\XiFun(E)}{\XiFun(0)}.
\]
\end{theorem}

% -------------------------------------


% ============================================================

% ============================================================

% [Section 6 Intro Modification]

\section{Numerical Signatures and Deep Space Phases}
We distinguish between the \textbf{Critical Coupling} ($\epsilon=2.5$), where the Berry phase locks to half-integer values, and the \textbf{Deep Space Regimes} ($\epsilon \gg 10$), where high-order resonances emerge.
\subsection{Phase I: Parametric Resonance ($\epsilon \approx 148$)}

At $\epsilon \approx 148.0$, we observe a hyper-excited state previously termed "The Monster." 
Crucially, this is not a numerical instability but a \textbf{high-order parametric resonance}. 
The ratio of the topological charge to the stiffness satisfies an integer harmonic condition:

\begin{equation}
\frac{\epsilon}{\mu} \approx \frac{148.0}{4.93} \approx 30.02.
\end{equation}

This confirms that the confinement potential (the "Cosmic Spine") vibrates in integer synchronization with the Berry-phase twist.

\subsection{Phase II: The Void ($\epsilon \approx 350$)}

At higher couplings ($\epsilon \approx 350$), destructive interference creates a "Silent Zone" (Fig.~\ref{fig:stokes}), analogous to Anderson localization, providing a stable regime for spectral rigidity.

% --- Figures ---

\begin{figure}[h]
\centering
\includegraphics[width=0.92\textwidth]{result.png}
\caption{Stokes phenomenon: turning-point trajectory in the complex plane (critical resonance highlighted).}
\label{fig:stokes}
\end{figure}

\begin{figure}[h]
\centering
\includegraphics[width=0.92\textwidth]{sff_analysis.png}
\caption{Spectral form factor showing a ramp/plateau structure (quantum chaos fingerprint) and a characteristic time scale.}
\label{fig:sff}
\end{figure}

\begin{figure}[h]
\centering
\includegraphics[width=0.92\textwidth]{berry_phase_maslov_correction.png}
\caption{Berry phase versus lattice strength $\epsilon$, illustrating locking behavior near $\epsilon=2.5$ (Maslov correction indicated).}
\label{fig:berry}
\end{figure}

\begin{figure}[h]
\centering
\includegraphics[width=0.92\textwidth]{gue_verification.png}
\caption{Level spacing statistics compared to GUE and Poisson benchmarks.}
\label{fig:gue}
\end{figure}

% ============================================================

\section{Conclusion}

We presented a unified operator framework in which arithmetic time-support $\log n$ and von Mangoldt weights arise as
exact distributional trace identities, while PT observability is enforced by an explicitly derived bounded metric.
Discrete spectra and determinants are defined via an IR topological regulator in log-space, and a canonical subtraction
principle fixes the normalized determinant to coincide with $\Xi(E)/\Xi(0)$.

The framework is intentionally modular: the kinematic arithmetic algebra is rigid, while physical realization can be pursued
in different platforms (non-Hermitian wave systems, driven circuits, engineered metamaterials) without altering the determinant-level structure.

% ============================================================

\appendix

\section{Parity sector and log map}
\label{app:parity_log}

\subsection{Odd parity and the origin}
The map $q=\log x$ requires $x>0$. We enforce this rigorously by restricting to the odd-parity sector
\[
\psi(-x)=-\psi(x),
\]
which implies $\psi(0)=0$ and allows a consistent half-line reduction.

\subsection{Unitary equivalence}
The map $\mathcal U$ in~\eqref{eq:U_log} is unitary from $L^2(\R_+,dx)$ to $L^2(\R,dq)$.
On this space, $\Hzero$ becomes the translation generator $-i\partial_q$.

\section{Derivation of the PT metric}
\label{app:pt_derivation}

Starting from~\eqref{eq:eta_ode},
\[
(\ln\eta)'=-2\lambda\Big(e^{-x^2}+\epsilon\frac{\sin x}{x}\Big),
\]
so
\[
\ln\eta(x)=-2\lambda\int_0^x e^{-u^2}\,du-2\lambda\epsilon\int_0^x \frac{\sin u}{u}\,du,
\]
giving~\eqref{eq:etaPT_closed}.

\section{IR regulator and determinants}
\label{app:ir_det}

On the twisted circle~\eqref{eq:twisted_bc}, the eigenvalues of $-i\partial_q$ are
\[
k_m=\frac{1}{2L}\big(2\pi m+\gamma_B\big),\qquad m\in\mathbb Z,
\]

so $\Htil_L^2$ has eigenvalues $k_m^2$. The determinant ratio~\eqref{eq:Delta_L} is then defined by standard zeta regularization.

\section{Canonical subtraction and stability of the prime term}
\label{app:canon}

Axiom~\ref{ax:canonicality}(C3) formalizes the stability of the atomic time-support:
counterterms may change only the smooth (archimedean/Weyl) part, not the distributional spikes at $t=\log n$.


% --- Bibliography ---
\begin{thebibliography}{99}

\bibitem{Atiyah1968}
M. F. Atiyah and I. M. Singer,
\textit{The Index of Elliptic Operators: I},
Ann. of Math. \textbf{87}, 484 (1968).

\bibitem{Berry1999}
M. V. Berry and J. P. Keating,
\textit{The Riemann Zeros and Eigenvalue Asymptotics},
SIAM Review \textbf{41}, 236 (1999).

\bibitem{Bender1998}
C. M. Bender and S. Boettcher,
\textit{Real Spectra in Non-Hermitian Hamiltonians Having PT Symmetry},
Phys. Rev. Lett. \textbf{80}, 5243 (1998).

\bibitem{Pavlov2025}
D. Seo and CosmosT,
\textit{The Riemann-Pavlov Equation: Topological Quantization via CP-Symmetry},
Preprint (2025).

\end{thebibliography}

\end{document}
